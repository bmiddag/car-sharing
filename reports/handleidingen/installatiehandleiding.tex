\documentclass[11pt,a4paper,oneside]{article}
\usepackage[dutch]{babel}
\usepackage{amsmath}
\usepackage{graphicx}
\usepackage{tikz}
\usepackage{fancyhdr}
\usepackage{dsfont}
\usepackage{parskip}
\usepackage{epstopdf}
\usepackage{listings}
\usepackage{color}
\definecolor{lichtgrijs}{gray}{0.95}
\usepackage{cite}
\usepackage[nottoc]{tocbibind}
\usepackage[T1]{fontenc}
\usepackage[light,math]{iwona}
\usepackage{pgfplots}
\pgfplotsset{compat=newest}
\usepackage{subfig}
\usepackage{textcomp}
\usetikzlibrary{arrows,automata}
\usepackage{float}
\usepackage{longtable}
\usepackage{titling,enumitem}
\usepackage{a4wide}
\usepackage{amssymb}
\usepackage{rotating}
\usepackage{listings}
\usepackage[top=1.1in, bottom=1.2in, left=1.1in, right=1.1in]{geometry}
\usepackage{array}
\usepackage{titling}
\usepackage{blindtext}
\usepackage{hyperref}
\usepackage{chngpage}
\usepackage{calc}
\definecolor{lightgray}{gray}{0.8}
\newcolumntype{L}{>{\raggedleft}p{0.50\textwidth}}
\newcolumntype{R}{p{0.8\textwidth}}
\newcommand\VRule{\color{lightgray}\vrule width 0.5pt}
\usepackage{color, colortbl}
\definecolor{Gray}{gray}{0.9}
\definecolor{dkgreen}{rgb}{0,0.6,0}
\definecolor{gray}{rgb}{0.5,0.5,0.5}
\definecolor{mauve}{rgb}{0.58,0,0.82}
\definecolor{ugentblue}{rgb}{0.05,0.18,0.37}
\usepackage{import}
\usepackage[T1]{fontenc}
\lstset{
language=BASH,
basicstyle=\scriptsize\ttfamily,
numbers=none,
numberstyle=\tiny\color{black},
firstnumber=1,
stepnumber=1,
numbersep=9pt,
backgroundcolor=\color{white},
showspaces=false,
showstringspaces=false,
showtabs=false,
frame=single,
tabsize=1,
captionpos=t,
title=\lstname,
breaklines=true,
breakatwhitespace=true,
keywordstyle=\color{blue},
commentstyle=\color{dkred}\textit,
stringstyle=\color{dkgreen},
escapeinside={\%*}{*)},
}


\begin{document}

\newgeometry{top=0.8cm, right=1.70cm, left=1.7cm}
\begin{titlepage}

\thispagestyle{fancy}
\fancyhf{}
\fancyfoot[L]{}
\begin{figure}[!ht]
  \begin{adjustwidth}{-\oddsidemargin-1in}{-\rightmargin}
    \centering
    \includegraphics[width=\paperwidth]{img/banner}
  \end{adjustwidth}
\end{figure}
\vspace{-0.2em}
\begin{center}
\vspace{5cm}
\Huge \textbf{Vakoverschrijdend Project: team Edran\\ Installatiehandleiding}\\
\vspace{6.0cm}
\large
\begin{tabular}{L! {} R}
& {\LARGE\bf Team Edran} \\
& \\
& {\bf Steven De Blieck} \\
& {\bf Laurens De Graeve} \\
& {\bf Bart Middag} \\
& {\bf Wouter Pinnoo} \\
& {\bf Robin Praet} \\
& {\bf Stijn Seghers} \\
& {\bf Wouter Termont} \\
& {\bf Gilles Vandewiele} \\
\end{tabular}
\end{center}
\end{titlepage}
\restoregeometry
\newpage

\fancyheadoffset[RO,LE]{0in}
\fancypagestyle{plain}{
\fancyhead[L]{Installatiehandleiding}
\fancyhead[R]{Team Edran}
\fancyfoot[L]{}
\fancyfoot[R]{\thepage}}

\fancyhead[L]{Installatiehandleiding}
\fancyhead[R]{Team Edran}
\fancyfoot[L]{}
\fancyfoot[C]{\thepage}
\pagestyle{fancy}

\section{Inleiding}
Deze handleiding beschrijft welke stappen moeten ondernomen worden om alles op de server in gereedheid te brengen voor de web-applicatie, en om de web-applicatie zelf te installeren.

\section{Databank}
Als databankserver gebruiken we MySQL. We voorzien scripts voor de installatie van de databank.

\textbf{Stap 1:} Installeer MySQL. Zie hiervoor de offici\"ele documentatie beschikbaar op \url{http://dev.mysql.com/doc/refman/5.1/en/installing.html}.\par
\textbf{Stap 2:} Stel een wachtwoord in voor de \verb|root|-user als je dit nog niet gedaan hebt tijdens de installatie van MySQL:
\vspace{-1.7em}\begin{lstlisting}
mysqladmin -u root password JOUWNIEUWWACHTWOORD
\end{lstlisting}\par
\textbf{Stap 3:} Maak een clone van onze GitHub server-repo:
\vspace{-1.7em}\begin{lstlisting}
git clone git@github.ugent.be:edran/server.git
\end{lstlisting}\par
\textbf{Stap 4:} In de repository is er een bestand \verb|database.conf| in de map \verb|configurations/|. Zet daar het wachtwoord van de \verb|root|-user in; in volgend formaat:
\vspace{-1.7em}\begin{lstlisting}
DBusr=root
DBpass=mynewpassword
DBsft=mysql
\end{lstlisting}\par
\textbf{Stap 5:} Ga naar de map \verb|DBscripts/| in de repository en voer daar het script \verb|setup.sh| uit met als parameter \verb|prod| (om de productie-databank te initialiseren):
\vspace{-1.7em}\begin{lstlisting}
cd server/DBscripts
./resetDB.sh prod
\end{lstlisting}\par
Proficiat! De databank is nu ge\"initialiseerd en klaar voor gebruik in de web-applicatie.

\section{Modules}
Onze modules (databank- en authenticatiemodule) worden onafhankelijk gemaakt van de webapplicatie, en moeten bijgevolg afzonderlijk gecompileerd worden.

\textbf{Stap 1:} Maak een clone van onze GitHub repository die de source code van de web-applicatie bevat:
\vspace{-1.7em}\begin{lstlisting}
git clone git@github.ugent.be:edran/autodelen.git
\end{lstlisting}\par
\textbf{Stap 2:} Voer het script \verb|compile_modules.sh| uit. Nu zitten in de map \verb|autodelen/lib/| de nieuwste versies (in \verb|jar|-formaat) van de modules.\par
De web-applicatie kan nu gebruik maken van de modules. Indien je het Play-framework ge\"installeerd hebt, kan je ook het commando \verb|play modules| gebruiken om de modules te compileren.

\section{Apache server}\label{sec:apache}
We installeren de play-webapplicatie op de locatie \verb|/prod| op de server. Als font-end webserver gebruiken we Apache2, en zal dienen voor de (reverse) proxy.\par
\textbf{Stap 1:} Maak een clone van onze GitHub repo die de configuraties van de server bevat:
\vspace{-1.7em}\begin{lstlisting}
git clone git@github.ugent.be:edran/server.git
\end{lstlisting}\par
\textbf{Stap 2:} Ga naar de \verb|configurations| map:
\vspace{-1.7em}\begin{lstlisting}
cd server/configurations
\end{lstlisting}\par
\textbf{Stap 3:} Pas in het bestand \verb|apache_configuration.sh| de \verb$SERVER_PATH$ variabele aan naar de map waar je de repository gecloned hebt. Bijvoorbeeld:
\vspace{-1.7em}\begin{lstlisting}
SERVER_PATH="/home/wouter/server"
\end{lstlisting}\par
\textbf{Stap 4:} Voer het shellscript \verb$apache_configuration.sh$ uit. \textbf{Let op:} hiervoor heb je sudo-rechten nodig.
\vspace{-1.7em}\begin{lstlisting}
sudo ./apache_configuration.sh
\end{lstlisting}\par
\textbf{Stap 5:} Proficiat! Je Apache2 web-server is nu ingesteld!

\section{Play applicatie}
\textbf{Stap 1:} Ga naar de server-repository die je in Sectie \ref{sec:apache} gecloned hebt. In de map \verb|play| vind je bestanden \verb|start.sh| en \verb|stop.sh|. Pas in die bestanden de variabelen \verb|PLAY_PATH| en \verb|SERVER_PATH| aan die het pad bevatten waar je de repositories 'autodelen', respectivelijk 'server' gecloned hebt. Bijvoorbeeld:
\vspace{-1.7em}\begin{lstlisting}
PLAY_PATH="/home/wouter/autodelen"
SERVER_PATH="/home/wouter/server"
\end{lstlisting}\par
\textbf{Stap 2:} Terug in de map waar je de server-repository gecloned hebt, vind je een shellscript \verb|play_deploy.sh|. Voer dit script uit om de 'autodelen' web-applicatie te compileren en op te zetten in productie modus. \textbf{Let op:} dit script gaat ervan uit dat de huidige gebruiker sudo-rechten heeft, met \verb|NOPASSWD|. Meer info hierover op \url{https://help.ubuntu.com/community/Sudoers#User_Specifications}.
\vspace{-1.7em}\begin{lstlisting}
./play_deploy.sh
\end{lstlisting}
Nu zal de play-web-applicatie gestart worden op de poort ($9000$) en prefix (\verb|/prod|) die ingesteld zijn in \verb|$SERVER_PATH/play/start.sh|.
\par


\end{document}
