\documentclass[11pt,a4paper,oneside]{article}
\usepackage[dutch]{babel}
\usepackage{amsmath}
\usepackage{graphicx}
\usepackage{tikz}
\usepackage{fancyhdr}
\usepackage{dsfont}
\usepackage{parskip}
\usepackage{epstopdf}
\usepackage{listings}
\usepackage{color}
\definecolor{lichtgrijs}{gray}{0.95}
\usepackage{cite}
\usepackage[nottoc]{tocbibind}
\usepackage[T1]{fontenc}
\usepackage[light,math]{iwona}
\usepackage{pgfplots}
\pgfplotsset{compat=newest}
\usepackage{subfig}
\usepackage{textcomp}
\usetikzlibrary{arrows,automata}
\usepackage{float}
\usepackage{longtable}
\usepackage{titling,enumitem}
\usepackage{a4wide}
\usepackage{amssymb}
\usepackage{rotating}
\usepackage{listings}
\usepackage[top=1.1in, bottom=1.2in, left=1.1in, right=1.1in]{geometry}
\usepackage{array}
\usepackage{titling}
\usepackage{blindtext}
\usepackage{hyperref}
\usepackage{chngpage}
\usepackage{calc}
\definecolor{lightgray}{gray}{0.8}
\newcolumntype{L}{>{\raggedleft}p{0.50\textwidth}}
\newcolumntype{R}{p{0.8\textwidth}}
\newcommand\VRule{\color{lightgray}\vrule width 0.5pt}
\usepackage{color, colortbl}
\definecolor{Gray}{gray}{0.9}
\definecolor{dkgreen}{rgb}{0,0.6,0}
\definecolor{gray}{rgb}{0.5,0.5,0.5}
\definecolor{mauve}{rgb}{0.58,0,0.82}
\definecolor{ugentblue}{rgb}{0.05,0.18,0.37}
\usepackage{import}
\usepackage[T1]{fontenc}
\lstset{
language=BASH,
basicstyle=\scriptsize\ttfamily,
numbers=none,
numberstyle=\tiny\color{black},
firstnumber=1,
stepnumber=1,
numbersep=9pt,
backgroundcolor=\color{white},
showspaces=false,
showstringspaces=false,
showtabs=false,
frame=single,
tabsize=1,
captionpos=t,
title=\lstname,
breaklines=true,
breakatwhitespace=true,
keywordstyle=\color{blue},
commentstyle=\color{dkred}\textit,
stringstyle=\color{dkgreen},
escapeinside={\%*}{*)},
}


\begin{document}

\newgeometry{top=0.8cm, right=1.70cm, left=1.7cm}
\begin{titlepage}

\thispagestyle{fancy}
\fancyhf{}
\fancyfoot[L]{}
\begin{figure}[!ht]
  \begin{adjustwidth}{-\oddsidemargin-1in}{-\rightmargin}
    \centering
    \includegraphics[width=\paperwidth]{img/banner}
  \end{adjustwidth}
\end{figure}
\vspace{-0.2em}
\begin{center}
\vspace{5cm}
\Huge \textbf{Vakoverschrijdend Project: team Edran\\ Ontwerp en technische beslissingen}\\
\vspace{6.0cm}
\large
\begin{tabular}{L! {} R}
& {\LARGE\bf Team Edran} \\
& \\
& {\bf Steven De Blieck} \\
& {\bf Laurens De Graeve} \\
& {\bf Bart Middag} \\
& {\bf Wouter Pinnoo} \\
& {\bf Robin Praet} \\
& {\bf Stijn Seghers} \\
& {\bf Wouter Termont} \\
& {\bf Gilles Vandewiele} \\
\end{tabular}
\end{center}
\end{titlepage}
\restoregeometry
\newpage

\fancyheadoffset[RO,LE]{0in}
\fancypagestyle{plain}{
\fancyhead[L]{Ontwerp en technische beslissingen}
\fancyhead[R]{Team Edran}
\fancyfoot[L]{}
\fancyfoot[R]{\thepage}}

\fancyhead[L]{Ontwerp en technische beslissingen}
\fancyhead[R]{Team Edran}
\fancyfoot[L]{}
\fancyfoot[C]{\thepage}
\pagestyle{fancy}

\section{Subsystemen}
Dit is een overzicht van de grote subsystemen waarin de applicatie kan opgedeeld worden:
\begin{itemize}
\item Databank
\item Mail
\item Infosessies
\item Reserveren
\end{itemize}

\section{Technische Beslissingen per subsysteem}
\subsection{Algemeen}
\begin{itemize}
\item Voor HTML/css hebben we gekozen voor Twitter Bootstrap. Dit wegens de mobile-first approach en de vele "snippets" die hier gratis voor te vinden zijn.
\end{itemize}
\subsection{Databank}
\begin{itemize}
\item DAO zo onafhankelijk mogelijk gemaakt van de rest van de applicatie
\item Referenties gebeuren met objecten ipv sleutels voor betere toegankelijkheid in de controllers
\item Timestamps gebruikt wegens de integratie met Java
\item Lossere Mysql integriteitsrestricties voor portabiliteit, extra controle moet dan in (form) validators gebeuren
\end{itemize}
\subsection{Mail}
\begin{itemize}
\item We hebben de Play "mailer" plug-in gekozen wegens zijn simpelheid en functies die voldoen aan onze vereisten.
\item De templates van de te verzenden e-mails worden in de Databank opgeslagen. Deze worden at runtime gecompileerd. Het voordeel is dat ze dus kunnen aangepast worden (vb. met bijkomstige variabelen).
\end{itemize}
\subsection{Infosessies}
\begin{itemize}
\item Geen noemenswaardige beslissingen
\end{itemize}
\subsection{Reserveren}
\begin{itemize}
\item Voor de logica van de kalender werd javascript gebruikt (public/reserve.js).
\item Ajax wordt gebruikt om op de client-side een xml op te vragen die gebruikt wordt om de weergave dynamisch aan te passen:
\begin{itemize}
\item Reservations.scala.xml: doel is weten welke dagen in de kalender rood/groen moeten gekleurd worden
\item Cars.scala.xml: doel is weten welke autos vrij zijn in de periode geselecteerd in de kalender
\end{itemize}
\end{itemize}


\section{Gebruikte architecturen}
\begin{itemize}
        \item Er wordt gebruikt gemaakt van \emph{Ajax} en \emph{JavaScript}.
        \item Het gebruikte dialect voor het databankbeheer is \emph{MySQL}.
\end{itemize}
 
\section{Programmeersconventies}
\begin{itemize}
        \item Methode- en veldnamen moeten in het Engels en camelcase.
\end{itemize}
 
\section{Afspraken over het versiesysteem}
\begin{itemize}
        \item Voor een nieuwe functie te implementeren wordt een nieuwe branch aangemaakt, deze wordt pas gemerged met de development server nadat deze af is.
\end{itemize}
 
\section{Voorschriften voor programmadocumentatie}
\begin{itemize}
        \item Methodes die niet duidelijk zijn bij naam alleen moeten gedocumenteerd worden.
        \item Bovenaan elke afgewerkte Java-klasse zetten we de edran signatuur (zie momenteel de DAO-klasses). Hierbovenop komt de datum van creatie en de naam van de persoon die klasse aanmaakt.
\end{itemize}
 
\section{Gebruikte IDE}
\begin{itemize}
        \item Iedereen gebruikt IntelliJ, die ondersteuning biedt aan het Play Framework, als programmeersomgeving.
\end{itemize}

\end{document}
