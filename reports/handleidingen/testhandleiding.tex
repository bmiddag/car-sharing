\documentclass[11pt,a4paper,oneside]{article}
\usepackage[dutch]{babel}
\usepackage{amsmath}
\usepackage{graphicx}
\usepackage{tikz}
\usepackage{fancyhdr}
\usepackage{dsfont}
\usepackage{parskip}
\usepackage{epstopdf}
\usepackage{listings}
\usepackage{hyperref}
\usepackage{color}
\definecolor{lichtgrijs}{gray}{0.95}
\usepackage{cite}
\usepackage[nottoc]{tocbibind}
\usepackage[T1]{fontenc}
\usepackage[light,math]{iwona}
\usepackage{pgfplots}
\pgfplotsset{compat=newest}
\usepackage{subfig}
\usepackage{textcomp}
\usetikzlibrary{arrows,automata}
\usepackage{float}
\usepackage{longtable}
\usepackage{titling,enumitem}
\usepackage{a4wide}
\usepackage{amssymb}
\usepackage{rotating}
\usepackage{listings}
\usepackage[top=1.1in, bottom=1.2in, left=1.1in, right=1.1in]{geometry}
\usepackage{array}
\usepackage{titling}
\usepackage{blindtext}
\usepackage{chngpage}
\usepackage{calc}
\definecolor{lightgray}{gray}{0.8}
\newcolumntype{L}{>{\raggedleft}p{0.50\textwidth}}
\newcolumntype{R}{p{0.8\textwidth}}
\newcommand\VRule{\color{lightgray}\vrule width 0.5pt}
\usepackage{color, colortbl}
\definecolor{Gray}{gray}{0.9}
\definecolor{dkgreen}{rgb}{0,0.6,0}
\definecolor{gray}{rgb}{0.5,0.5,0.5}
\definecolor{mauve}{rgb}{0.58,0,0.82}
\definecolor{ugentblue}{rgb}{0.05,0.18,0.37}
\usepackage{import}
\usepackage[T1]{fontenc}
\begin{document}

\newgeometry{top=0.8cm, right=1.70cm, left=1.7cm}
\begin{titlepage}

\thispagestyle{fancy}
\fancyhf{}
\fancyfoot[L]{}
\begin{figure}[!ht]
  \begin{adjustwidth}{-\oddsidemargin-1in}{-\rightmargin}
    \centering
    \includegraphics[width=\paperwidth]{img/banner}
  \end{adjustwidth}
\end{figure}

\vspace{-0.2em}
\begin{center}
\vspace{5cm}
\Huge \textbf{Vakoverschrijdend Project: team Edran\\ Testhandleiding}\\
\vspace{6.0cm}
\large
\begin{tabular}{L! {} R}
& {\LARGE\bf Team Edran} \\
& \\
& {\bf Steven De Blieck} \\
& {\bf Laurens De Graeve} \\
& {\bf Bart Middag} \\
& {\bf Wouter Pinnoo} \\
& {\bf Robin Praet} \\
& {\bf Stijn Seghers} \\
& {\bf Wouter Termont} \\
& {\bf Gilles Vandewiele} \\
\end{tabular}
\end{center}
\end{titlepage}
\restoregeometry
\newpage

\fancyheadoffset[RO,LE]{0in}
\fancypagestyle{plain}{
\fancyhead[L]{Testhandleiding}
\fancyhead[R]{Team Edran}
\fancyfoot[L]{}
\fancyfoot[R]{\thepage}}

\fancyhead[L]{Testhandleiding}
\fancyhead[R]{Team Edran}
\fancyfoot[L]{}
\fancyfoot[C]{\thepage}
\pagestyle{fancy}
\tableofcontents

\section{Inleiding}
Voor onze testen maakten we gebruik van een Jenkins-platform. Deze is terug te vinden op \url{http://edran.ugent.be/jenkins/}. We maakten een GitHub webhook op github.ugent.be die bij elke commit naar de \verb|master| of \verb|development| branch een trigger stuurde naar Jenkins.
Jenkins doet na het ontvangen van een trigger het volgende:
\begin{itemize}
\item het commando \verb|clean compile| uitvoeren in \verb|sbt|: zo worden compileerfouten in de web-applicatie zelf ontdekt.
\item tests van onze eigen modules uitvoeren. Deze wordt gedefinieerd in shell scripts en vindt u terug op \url{http://edran.ugent.be/jenkins/job/autodelen/configure} bij 'bouwstappen' (meer hierover in Sectie \ref{sec:modules}).
\item alle JUnit tests uitvoeren (meer hierover in Sectie \ref{sec:junit}).
\end{itemize}
Als het uitvoeren van \'e\'en van bovenstaande operaties niet succesvol was, gebeurt het volgende:
\begin{itemize}
\item de mislukte bouwpoging wordt in rekening gehouden met de statistieken die het Jenkins-platform zelf bijhoudt
\item op het \verb|README.md|-bestand op de repository toont het logo van Jenkins aan dat de bouwpoging mislukt was
\item het auteur van de commit die voor een mislukte bouwpoging zorgde, wordt via e-mail op de hoogte gebracht.
\end{itemize} 


\section{Eigen tests voor modules}\label{sec:modules}
We maakten onze modules (bv. DAO's) volledig onafhankelijk van de webapplicatie. Zo is deze herbruikbaar in andere projecten. We compileren de modules als \verb|.jar|-bestanden en gebruiken ze als library in de webapplicatie.\par
Om het gebruikte \verb|.jar|-bestand up-to-date te houden, en om compileerfouten uit de modules snel te herkennen, worden de modules door Jenkins gecompileerd, verpakt in \verb|.jar|'s en gepushed naar de repository door Jenkins.

\section{Tests uitvoeren}\label{sec:junit}
Om tests uit te voeren ga je naar de directory \emph{autodelen}. Hierin voer je het script \emph{test.sh} uit. (\textbf{./test.sh}) \\
\textbf{BELANGRIJK: aan de volgende vereisten moeten voldaan zijn:} 
\begin{itemize}
\item	firefox moet op de computer staan
\item	de directory \emph{server} en \emph{autodelen} moeten in eenzelfde map aanwezig zijn
\end{itemize} 
Indien het script ./test.sh niet werkt, dan kan je handmatig de databank resetten, het commando \emph{play run} in een aparte terminal uitvoeren en daarna play test.
\end{document}
