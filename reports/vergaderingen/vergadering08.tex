\documentclass[11pt,a4paper,oneside]{article}

\usepackage[dutch]{babel}
\usepackage{fullpage}
\usepackage{tabularx}
\usepackage{courier}
\usepackage{tikz}
\usepackage{graphicx}
\usepackage{algorithm}
\usepackage{algpseudocode}
\usepackage{tkz-graph}
\usepackage[light,math]{iwona}
\usepackage[T1]{fontenc}
\usepackage{float}
	\floatname{algorithm}{Algoritme}
\usepackage{hyperref}
\usepackage{wrapfig}
\usepackage{listings, multicol}
\usepackage{color}
	\definecolor{lichtgrijs}{rgb}{0.98, 0.98, 0.98}
	\definecolor{dkgreen}{rgb}{0,0.4,0}
	\definecolor{dkred}{rgb}{0.45,0,0}
	\definecolor{lichtgrijs}{gray}{0.95}
\usepackage{longtable}
\usepackage{paralist}
\usepackage[top=.7in, bottom=.9in, left=.8in, right=.8in]{geometry}
\usepackage{import}
\usepackage{fmtcount}
%\usepackage{gensymb} % throws errors: 'not defining \perthousand' and 'noet defining \micro'
\usepackage{array}
	\newcolumntype{L}[1]{>{\raggedright\let\newline\\\arraybackslash\hspace{0pt}}m{#1}}
	\newcolumntype{C}[1]{>{\centering\let\newline\\\arraybackslash\hspace{0pt}}m{#1}}
	\newcolumntype{R}[1]{>{\raggedleft\let\newline\\\arraybackslash\hspace{0pt}}m{#1}}
\usepackage{parskip}
\usepackage{multirow}
\usepackage{titling}
\usepackage{fancyhdr}
\usepackage{mathtools}
	\DeclarePairedDelimiter{\ceil}{\lceil}{\rceil}
\usepackage{import}
\usepackage{blindtext}
\usepackage{amssymb}
\usepackage{amsmath}
\usepackage{caption}
\usepackage{subcaption}
\usepackage{paralist}
	\let\itemize\compactitem
	\let\enditemize\endcompactitem
	\let\enumerate\compactenum
	\let\endenumerate\endcompactenum
	\let\description\compactdesc
	\let\enddescription\endcompactdesc
	\pltopsep=\medskipamount
	\plitemsep=2pt
	\plparsep=2pt

\newcommand{\todo}[1]{{\color{red} \textbf{#1}}}
\renewenvironment{itemize}[1]{\begin{compactitem}#1}{\end{compactitem}}
\renewenvironment{enumerate}[1]{\begin{compactenum}#1}{\end{compactenum}}
\renewenvironment{description}[0]{\begin{compactdesc}}{\end{compactdesc}}

\lstset{
language=JAVA,
basicstyle=\scriptsize\ttfamily,
numbers=none,
numberstyle=\tiny\color{black},
firstnumber=1,
stepnumber=1,
numbersep=9pt,
backgroundcolor=\color{white},
showspaces=false,
showstringspaces=false,
showtabs=false,
frame=single,
tabsize=1,
captionpos=t,
title=\lstname,
breaklines=true,
breakatwhitespace=true,
keywordstyle=\color{blue},
commentstyle=\color{dkred}\textit,
stringstyle=\color{dkgreen},
escapeinside={\%*}{*)},
}

\title{Vergadering 8}
\author{Team Edran}
\date{24 april 2014}

\begin{document}

\maketitle

\textbf{Plaats \& tijd:}
Het kot van Gilles; 16u00-17u00\\
\textbf{Aanwezig:} iedereen\par
\textbf{Agendapunten:}
\begin{itemize}
\item \textbf{Overlopen wat er gedaan is:}\\
\begin{itemize}
\item Tests infosessies: klaar, maar nog bugs
\item Autogebonden gegevens: klaar
\item Bewijsmateriaal uploaden: bijna klaar
\item Verifi\"eren van bewijsmateriaal: view is er al, maar logica moet nog gekoppeld worden
\item Notificaties: nog wat werk aan
\item Beheer notificaties: nog wat werk aan
\item Tijdelijk blokkeren gebruiker: klaar, maar e-mails moeten nog gestuurd worden en permissies moeten nog toegevoegd worden
\end{itemize}

\item \textbf{Notificaties:\\}
Voor deze mijlpaal zullen we de notificaties hardcoderen in het systeem. Later zullen we ze instelbaar maken.\\
We moeten een lijst maken van welke notificaties we moeten maken. Deze lijst komt in een Github issue of in een Google Drive-document.\\
We zullen alvast een notificatie moeten sturen voor alles wat ingediend en goed/afgekeurd wordt, zowel aan de kant van de autolener als aan die van de autobeheerder.

\item \textbf{Schadedossiers:\\}
De rest van de implementatie van schadedossiers wordt verplaatst naar de volgende mijlpaal.

\item \textbf{Beheer tabbladen:\\}
De beheerspagina moet herzien worden: tabbladen zouden andere namen moeten krijgen en sommige mogen zelfs weg.

\item \textbf{Zones:\\}
Het is de bedoeling dat de beheerder kan kiezen van welke zone een gebruiker deel uitmaakt.\\
Zones laten we voor de volgende mijlpaal. Het is optioneel, dus het krijgt een iets lagere prioriteit. We laten zones dan ook weg uit het filtersysteem voor deze mijlpaal.

\item \textbf{Templates:\\}
Aan de beheerskant moet er nog een keuze zijn welke templates gebruikt worden voor wat. Gekozen templates mogen dan ook niet verwijderd worden.\\
Om de mails die met templates werken te versturen, is er al een parser van Bart beschikbaar, deze moet gewoon gebruikt worden.

\item \textbf{Ritgegevens:\\}
Invullen van ritgegevens moet gesynchroniseerd worden met een klok. De gebruiker zou niet zelf een nieuw ritgegeven moeten maken, het systeem zou dit voor hem moeten doen.\\
Er zijn ook nog veel bugs in ritgegevens.

\item \textbf{Internettechnologie:\\}
Het is afgeraakt bij iedereen!

\item \textbf{Taken tegen het einde van deze mijlpaal:\\}
Iedereen werkt af waar hij nog mee bezig was.
\begin{itemize}
\item Rapport genereren: Robin
\item Machtigingen toevoegen: Steven
\item Bewijsmateriaal: Wouter P.
\item Notificaties: Stijn en Gilles
\item Bewijsmateriaal verifieren: Laurens
\item Databank: Wouter T.
\item Auto toevoegen aan systeem door autobeheerder + als beheerder goedkeuren/afkeuren: Bart
\end{itemize}

\par\item \textbf{Volgende vergadering} \\
Maandag 28 april om 14u30, op Edran HQ - het kot van Laurens.\\
De volgende vergadering zouden we best nog eens brainstormen over wat er op het dashboard moet komen, de lay-out, welke statistieken we zullen oplijsten...\\
We zouden ook best features zoeken die we nog niet ge\"implementeerd hebben, zoals een zip-backup maken van de databank (dit is al ongeveer ge\"implementeerd - er kunnen al csv-bestanden van de data gemaakt worden.\\
We maken dan ook best voor elke nog te implementeren feature een Github issue.


\end{itemize}

\end{document}
