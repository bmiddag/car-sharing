\documentclass[11pt,a4paper,oneside]{article}

\usepackage[dutch]{babel}
\usepackage{fullpage}
\usepackage{tabularx}
\usepackage{courier}
\usepackage{tikz}
\usepackage{graphicx}
\usepackage{algorithm}
\usepackage{algpseudocode}
\usepackage{tkz-graph}
\usepackage[light,math]{iwona}
\usepackage[T1]{fontenc}
\usepackage{float}
	\floatname{algorithm}{Algoritme}
\usepackage{hyperref}
\usepackage{wrapfig}
\usepackage{listings, multicol}
\usepackage{color}
	\definecolor{lichtgrijs}{rgb}{0.98, 0.98, 0.98}
	\definecolor{dkgreen}{rgb}{0,0.4,0}
	\definecolor{dkred}{rgb}{0.45,0,0}
	\definecolor{lichtgrijs}{gray}{0.95}
\usepackage{longtable}
\usepackage{paralist}
\usepackage[top=.7in, bottom=.9in, left=.8in, right=.8in]{geometry}
\usepackage{import}
\usepackage{fmtcount}
%\usepackage{gensymb} % throws errors: 'not defining \perthousand' and 'noet defining \micro'
\usepackage{array}
	\newcolumntype{L}[1]{>{\raggedright\let\newline\\\arraybackslash\hspace{0pt}}m{#1}}
	\newcolumntype{C}[1]{>{\centering\let\newline\\\arraybackslash\hspace{0pt}}m{#1}}
	\newcolumntype{R}[1]{>{\raggedleft\let\newline\\\arraybackslash\hspace{0pt}}m{#1}}
\usepackage{parskip}
\usepackage{multirow}
\usepackage{titling}
\usepackage{fancyhdr}
\usepackage{mathtools}
	\DeclarePairedDelimiter{\ceil}{\lceil}{\rceil}
\usepackage{import}
\usepackage{blindtext}
\usepackage{amssymb}
\usepackage{amsmath}
\usepackage{caption}
\usepackage{subcaption}
\usepackage{paralist}
	\let\itemize\compactitem
	\let\enditemize\endcompactitem
	\let\enumerate\compactenum
	\let\endenumerate\endcompactenum
	\let\description\compactdesc
	\let\enddescription\endcompactdesc
	\pltopsep=\medskipamount
	\plitemsep=2pt
	\plparsep=2pt

\newcommand{\todo}[1]{{\color{red} \textbf{#1}}}
\renewenvironment{itemize}[1]{\begin{compactitem}#1}{\end{compactitem}}
\renewenvironment{enumerate}[1]{\begin{compactenum}#1}{\end{compactenum}}
\renewenvironment{description}[0]{\begin{compactdesc}}{\end{compactdesc}}

\lstset{
language=JAVA,
basicstyle=\scriptsize\ttfamily,
numbers=none,
numberstyle=\tiny\color{black},
firstnumber=1,
stepnumber=1,
numbersep=9pt,
backgroundcolor=\color{white},
showspaces=false,
showstringspaces=false,
showtabs=false,
frame=single,
tabsize=1,
captionpos=t,
title=\lstname,
breaklines=true,
breakatwhitespace=true,
keywordstyle=\color{blue},
commentstyle=\color{dkred}\textit,
stringstyle=\color{dkgreen},
escapeinside={\%*}{*)},
}

\title{Vergadering 3}
\author{Team Edran}
\date{4 maart 2014}

\begin{document}

\maketitle

\textbf{Plaats \& tijd:}
Campus Sterre, S9, Zeus-kelder; 14u20-14u55.

\textbf{Aanwezig:}
Iedereen

\textbf{Agendapunten:}

\begin{itemize}
    \item \textbf{Gedaan werk:}
        \begin{itemize}
            \item Laurens heeft verder gewerkt aan de database.
            \item Bart heeft de infosessies-view volledig afgewerkt.
            \item Gilles is voor het autobehoor begonnen met het linken van de
                view en de controller.
            \item Steven heeft de use cases en mock-ups afgewerkt.
            \item Wouter Termont heeft foutjes in de database verbeterd en
                teamleden geholpen met vragen rond de database.
            \item Stijn is begonnen met de integratie tussen de
                reservatiecontroller en de database. Het testen was nog niet
                gelukt door een gebrek aan een lokale database. Het idee wordt
                geopperd om zo snel mogelijk een lokale database draaiende te
                krijgen bij iedereen.
            \item Wouter Pinnoo heeft de Apache-server nog verder
                geconfigureerd, onder andere om twee versies van het Play!
                Framework naast elkaar draaiende te krijgen. Daarnaast heeft
                hij ook gewerkt aan de reservatieview.
        \end{itemize}

    \item \textbf{Formaat handleidingen:}
        De vraag werd gesteld of de handleidingen in Markdown in de
        GitHub-wiki, dan wel in Latex zouden moeten worden gemaakt.

        In de wiki kunnen snel aanpassingen gemaakt worden. De vraag wordt
        gesteld of dit makkelijk om te zetten is naar pdf. Aangezien hier wel
        degelijk tools voor bestaan, is iedereen er mee akkoord om Markdown te
        gebruiken.

    \item \textbf{Patroon voor DataAccessProvider en andere modellen:}
        Wat is het meest aangewezen patroon voor de DataAccessProvider en
        andere dergelijke modellen, een singleton of statische methoden?

        De algemene consensus neigt naar het gebruik van singletons.

    \item \textbf{Afspraak i.v.m.\ branching in Git:}
        In het huidige model zijn er slechts twee branches: \emph{master} en
        \emph{development}. Iedereen pusht op de development-branch en voor het
        effectief deployen wordt er gepusht naar de master-branch.

        Om vaak optrendende \emph{merge conflicts} te vermijden wordt het
        volgende model voorgesteld. Iedereen gebruikt een eigen
        development-branch en op een volgende vergadering worden alle aparte
        development-branchen gemerget in de algemene development-branch. De
        functies van de master- en development-branch blijven behouden.

        Iedereen is akkoord met dit nieuwe model.

    \item \textbf{Werkverdeling hackaton:}
        \begin{itemize}
            \item Laurens zal het script dat de database met testdata vult afwerken.
            \item Wouter Termont zal de toegang tot de database abstraheren
                a.d.h.v.\ een propertiesbestand, de primaire sleutel voor
                reservaties aanpasen en een functie toevoegen om alle auto's op
                te vragen.
            \item Bart en Robin zullen verder werken aan het infosessiebeheer.
            \item Voor het reserveren van ritten zal Gilles verder werken aan
                het goed- en afkeuren ervan, Wouter Pinnoo aan de
                ritgegevensview en Stijn aan het kalendergedeelte.
            \item Steven zal aan het email-beheer verder werken.
        \end{itemize}

\end{itemize}

\end{document}
