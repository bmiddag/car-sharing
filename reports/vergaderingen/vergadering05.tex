\documentclass[11pt,a4paper,oneside]{article}

\usepackage[dutch]{babel}
\usepackage{fullpage}
\usepackage{tabularx}
\usepackage{courier}
\usepackage{tikz}
\usepackage{graphicx}
\usepackage{algorithm}
\usepackage{algpseudocode}
\usepackage{tkz-graph}
\usepackage[light,math]{iwona}
\usepackage[T1]{fontenc}
\usepackage{float}
	\floatname{algorithm}{Algoritme}
\usepackage{hyperref}
\usepackage{wrapfig}
\usepackage{listings, multicol}
\usepackage{color}
	\definecolor{lichtgrijs}{rgb}{0.98, 0.98, 0.98}
	\definecolor{dkgreen}{rgb}{0,0.4,0}
	\definecolor{dkred}{rgb}{0.45,0,0}
	\definecolor{lichtgrijs}{gray}{0.95}
\usepackage{longtable}
\usepackage{paralist}
\usepackage[top=.7in, bottom=.9in, left=.8in, right=.8in]{geometry}
\usepackage{import}
\usepackage{fmtcount}
%\usepackage{gensymb} % throws errors: 'not defining \perthousand' and 'noet defining \micro'
\usepackage{array}
	\newcolumntype{L}[1]{>{\raggedright\let\newline\\\arraybackslash\hspace{0pt}}m{#1}}
	\newcolumntype{C}[1]{>{\centering\let\newline\\\arraybackslash\hspace{0pt}}m{#1}}
	\newcolumntype{R}[1]{>{\raggedleft\let\newline\\\arraybackslash\hspace{0pt}}m{#1}}
\usepackage{parskip}
\usepackage{multirow}
\usepackage{titling}
\usepackage{fancyhdr}
\usepackage{mathtools}
	\DeclarePairedDelimiter{\ceil}{\lceil}{\rceil}
\usepackage{import}
\usepackage{blindtext}
\usepackage{amssymb}
\usepackage{amsmath}
\usepackage{caption}
\usepackage{subcaption}
\usepackage{paralist}
	\let\itemize\compactitem
	\let\enditemize\endcompactitem
	\let\enumerate\compactenum
	\let\endenumerate\endcompactenum
	\let\description\compactdesc
	\let\enddescription\endcompactdesc
	\pltopsep=\medskipamount
	\plitemsep=2pt
	\plparsep=2pt

\newcommand{\todo}[1]{{\color{red} \textbf{#1}}}
\renewenvironment{itemize}[1]{\begin{compactitem}#1}{\end{compactitem}}
\renewenvironment{enumerate}[1]{\begin{compactenum}#1}{\end{compactenum}}
\renewenvironment{description}[0]{\begin{compactdesc}}{\end{compactdesc}}

\lstset{
language=JAVA,
basicstyle=\scriptsize\ttfamily,
numbers=none,
numberstyle=\tiny\color{black},
firstnumber=1,
stepnumber=1,
numbersep=9pt,
backgroundcolor=\color{white},
showspaces=false,
showstringspaces=false,
showtabs=false,
frame=single,
tabsize=1,
captionpos=t,
title=\lstname,
breaklines=true,
breakatwhitespace=true,
keywordstyle=\color{blue},
commentstyle=\color{dkred}\textit,
stringstyle=\color{dkgreen},
escapeinside={\%*}{*)},
}

\title{Vergadering 5}
\author{Team Edran}
\date{18 maart 2014}

\begin{document}

	\begin{itemize}
		
		\item Vooruitgang werk:
        \begin{itemize}
        	\item Steven
              \begin{itemize}
                  \item rapport samen met gilles
                  \item DeadBolt: beginselen (het zal wel lukken)
                  \item documentatie
              \end{itemize}
           	\item Bart
              \begin{itemize}
                  \item documentatie
              \end{itemize}
            \item Stijn
              \begin{itemize}
                  \item documentatie
              \end{itemize}
            \item Wouter P
              \begin{itemize}
                  \item Selenium op de server fixen
                  \item documentatie
                  \item test geschreven
                  \item view+model+controller gemaakt voor ritgegevens aan te passen
              \end{itemize}
            \item Wouter T
              \begin{itemize}
                  \item documentatie
                  \item refactoring: nulls teruggeven in plaats van errors
              \end{itemize}
            \item Gilles
              \begin{itemize}
                  \item documentatie
                  \item rapport
                  \item view+controller+model om ritgegevens goed/af te keuren
              \end{itemize}
            \item Laurens
              \begin{itemize}
                  \item documentatie
              \end{itemize}
            \item Robin
              \begin{itemize}
                  \item form validation
                  \item documentatie
              \end{itemize}
        \end{itemize}
        \textbf{Noot: er is wel veel voortuitgang in het practicum 01 van itech!}    
		
		\item BELANGRIJK: Als er nieuwe methodes aan de DAO's toegevoegd worden, moeten hier ook tests voor geschreven worden (liefst zelf, desnoods zeggen tegen Laurens/Wouter T)
		
		\item Normalisatie van de DB: EER-diagram was reeds genormaliseerd. \textbf{Rides} moet wel nog genormaliseerd worden.	
        
        \item JOIN-statements in de DAO's om de databank efficienter te gebruiken: Wouter Termont kijkt ervoor. 
        
        \item Login-systeem tegen volgende week maandag ge\"{i}mplementeerd.
        
        \item 	\begin{itemize}
        
        				\item Wouter P: ritgegevens toevoegen moet nog gebeuren
                    
                    \item Stijn: zie vorige vergadering
                    
                    \item Wouter T: SQL stmts moeten nog gebeuren, EER normalisatie (vooral voor Rides)
                    
                    \item Laurens: inlogsysteem (tegen nu maandag zeker af)
                    
                    \item Steven: zie vorige vergadering
                    
                    \item Robin: autogebonden gegevens ingeven
                    
                    \item Bart: e-mailtemplates afwerken, Facturisatie model
                    
                    \item Gilles: e-mailtemplates afwerken, testen schrijven (ook voor nieuw toegevoegde methodes aan DAO's)
        		\end{itemize}
        		
       \item Selenium tests: gebruik nooit \emph{driver.findElement(By...)} maar wel \emph{WebDriverWait wait = new WebDriverWait(driver, 5); wait.until(ExpectedConditions.visibilityOfElementLocated(By...)}
       
       \item \textbf{WOENSDAG OM 12.00 IS IEDEREEN KLAAR MET ZIJN WERK}
       
       \item Gilles stuurt mail naar alle drie de profs.
       
       \item Volgende vergadering: woensdagochtend om 10u, S2 bib
		
	\end{itemize}

\end{document}
