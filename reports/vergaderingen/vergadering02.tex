\documentclass[11pt,a4paper,oneside]{article}

\usepackage[dutch]{babel}
\usepackage{fullpage}
\usepackage{tabularx}
\usepackage{courier}
\usepackage{tikz}
\usepackage{graphicx}
\usepackage{algorithm}
\usepackage{algpseudocode}
\usepackage{tkz-graph}
\usepackage[light,math]{iwona}
\usepackage[T1]{fontenc}
\usepackage{float}
	\floatname{algorithm}{Algoritme}
\usepackage{hyperref}
\usepackage{wrapfig}
\usepackage{listings, multicol}
\usepackage{color}
	\definecolor{lichtgrijs}{rgb}{0.98, 0.98, 0.98}
	\definecolor{dkgreen}{rgb}{0,0.4,0}
	\definecolor{dkred}{rgb}{0.45,0,0}
	\definecolor{lichtgrijs}{gray}{0.95}
\usepackage{longtable}
\usepackage{paralist}
\usepackage[top=.7in, bottom=.9in, left=.8in, right=.8in]{geometry}
\usepackage{import}
\usepackage{fmtcount}
%\usepackage{gensymb} % throws errors: 'not defining \perthousand' and 'noet defining \micro'
\usepackage{array}
	\newcolumntype{L}[1]{>{\raggedright\let\newline\\\arraybackslash\hspace{0pt}}m{#1}}
	\newcolumntype{C}[1]{>{\centering\let\newline\\\arraybackslash\hspace{0pt}}m{#1}}
	\newcolumntype{R}[1]{>{\raggedleft\let\newline\\\arraybackslash\hspace{0pt}}m{#1}}
\usepackage{parskip}
\usepackage{multirow}
\usepackage{titling}
\usepackage{fancyhdr}
\usepackage{mathtools}
	\DeclarePairedDelimiter{\ceil}{\lceil}{\rceil}
\usepackage{import}
\usepackage{blindtext}
\usepackage{amssymb}
\usepackage{amsmath}
\usepackage{caption}
\usepackage{subcaption}
\usepackage{paralist}
	\let\itemize\compactitem
	\let\enditemize\endcompactitem
	\let\enumerate\compactenum
	\let\endenumerate\endcompactenum
	\let\description\compactdesc
	\let\enddescription\endcompactdesc
	\pltopsep=\medskipamount
	\plitemsep=2pt
	\plparsep=2pt

\newcommand{\todo}[1]{{\color{red} \textbf{#1}}}
\renewenvironment{itemize}[1]{\begin{compactitem}#1}{\end{compactitem}}
\renewenvironment{enumerate}[1]{\begin{compactenum}#1}{\end{compactenum}}
\renewenvironment{description}[0]{\begin{compactdesc}}{\end{compactdesc}}

\lstset{
language=JAVA,
basicstyle=\scriptsize\ttfamily,
numbers=none,
numberstyle=\tiny\color{black},
firstnumber=1,
stepnumber=1,
numbersep=9pt,
backgroundcolor=\color{white},
showspaces=false,
showstringspaces=false,
showtabs=false,
frame=single,
tabsize=1,
captionpos=t,
title=\lstname,
breaklines=true,
breakatwhitespace=true,
keywordstyle=\color{blue},
commentstyle=\color{dkred}\textit,
stringstyle=\color{dkgreen},
escapeinside={\%*}{*)},
}

\title{Vergadering 2}
\author{Team Edran}
\date{21 februari 2014}

\begin{document}

\maketitle

\textbf{Plaats \& tijd:}
Campus Sterre, S2, Bibliotheek; 13u15-14u15\\
\textbf{Aanwezig:} iedereen\par
\textbf{Agendapunten:}
\begin{itemize}
\item \textbf{Overlopen van noodzakelijke dingen voor iedereen: }\\
GUI beginnen implementeren zodat we snel aan het echte programmeerwerk kunnen beginnen.
Rows en columns zijn belangrijk zodat de applicatie responsief is.
Voor de database te vullen, een script uitvoeren.

\par\item \textbf{Overlopen van Jenkins:}\\ \todo{??}
\par\item \textbf{Bespreking rapport:}\\
Mock-ups moeten in het rapport vermeld worden. Best niet allemaal, anders wordt het vrij onmogelijk alles af te krijgen. Best use-case eens nakijken iedereen, zodat vergeten use-cases kunnen aangevuld worden.


\par\item \textbf{Belangrijke puntjes:}\\
Bij de databank moet een filtersysteem bijkomen.
'Ik ben aanwezig' zorgt voor onenigheid, moet nog eens herbekeken worden.
Enums worden bewust vermeden, deze zouden voor grote aanpassingen zorgen later.
Plugin bootstrap nog eens bekijken, of deze wel dan niet zal gebruikt worden.
Periode moet opgegeven worden bij het reserveren van een auto, anders heb je randgevallen. 

\par\item \textbf{Werkverdeling:} \\
Met veel aan de GUI beginnen, zodat de flow duidelijk wordt. Eerst de Views, dan pas de Controllers.
Wouter, Bart, Steven, Gilles zullen de GUI's doen.\par
Reserveren van rit en een rit zoeken: Wouter (delen-reserveren) \\
Delen van de ritgegevens: Gilles \\
Infosessies: Bart \\
Base template: Steven \\
Steven en Gilles mockups nog verder afwerken.

Stijn en Robin: kalender verder afwerken
\par\item \textbf{Rapport:} \\
Gilles zorgt voor het rapport. Opmerkingen daarna moeten vermeld worden.

\par\item \textbf{Volgende vergadering} \\
Volgende vergadering vindt plaats op woensdagnamiddag om 16:15.

\end{itemize}

\end{document}
