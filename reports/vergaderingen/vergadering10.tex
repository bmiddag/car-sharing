\documentclass[11pt,a4paper,oneside]{article}

\usepackage[dutch]{babel}
\usepackage{fullpage}
\usepackage{tabularx}
\usepackage{courier}
\usepackage{tikz}
\usepackage{graphicx}
\usepackage{hyperref}
\usepackage{wrapfig}
\usepackage{listings, multicol}
\usepackage{color}
	\definecolor{lichtgrijs}{rgb}{0.98, 0.98, 0.98}
	\definecolor{dkgreen}{rgb}{0,0.4,0}
	\definecolor{dkred}{rgb}{0.45,0,0}
	\definecolor{lichtgrijs}{gray}{0.95}
\usepackage{longtable}
\usepackage[top=.7in, bottom=.9in, left=.8in, right=.8in]{geometry}
\usepackage{import}
\usepackage{fmtcount}
\usepackage{parskip}
\usepackage{multirow}
\usepackage{titling}
\usepackage{fancyhdr}
\usepackage{mathtools}
	\DeclarePairedDelimiter{\ceil}{\lceil}{\rceil}
\usepackage{import}
\usepackage{blindtext}
\usepackage{amssymb}
\usepackage{amsmath}
\usepackage{caption}
\usepackage{subcaption}
\usepackage{paralist}
	\let\itemize\compactitem
	\let\enditemize\endcompactitem
	\let\enumerate\compactenum
	\let\endenumerate\endcompactenum
	\let\description\compactdesc
	\let\enddescription\endcompactdesc
	\pltopsep=\medskipamount
	\plitemsep=2pt
	\plparsep=2pt

\newcommand{\todo}[1]{{\color{red} \textbf{#1}}}
\renewenvironment{itemize}[1]{\begin{compactitem}#1}{\end{compactitem}}
\renewenvironment{enumerate}[1]{\begin{compactenum}#1}{\end{compactenum}}
\renewenvironment{description}[0]{\begin{compactdesc}}{\end{compactdesc}}


\title{Vergadering 10}
\author{Team Edran}
\date{5 mei 2014}

\begin{document}

\maketitle

\textbf{Plaats \& tijd:}
S9 Vergaderlokaal; 10u00-11u00\par
\textbf{Aanwezig:} iedereen \\
\textbf{Agendapunten:}
\begin{itemize}
\item \textbf{Vooruitgang:}\\
\begin{itemize}
\item Steven: rapport + mockup dashboard
\item Laurens: afwerken van meerdere issues
\item Bart: registratieprocedure verbeteren
\item Robin: remindermails
\item Stijn: legende kalender, omzetten javascript naar jQuery, verminderen van ongelezen notificaties na lezen
\item Waldo: getters in de DB generisch maken, bugfixes, Date-formaat van MySQL fixen, filters voor elke tabel
\item Gilles: volgende/vorige autolener, pagina voor auto's goed/af te keuren, grafieken meegeven met statistieken
\item Wouter: vroeger gemaakte pagina's terug laten werken, HTTPS, andere issues afwerken
\end{itemize}

\item \textbf{Dashboard overlopen}

\item \textbf{Nieuwe criteria (rookvrij, huisdieren toegelaten?, ...) worden toegevoegd in het systeem}

\item \textbf{Tegen einde van deze week (Zondag 11/05 00:00): iedereen fixed zijn eigen tests en documenteert zijn code}

\item \textbf{Robin zal de handleiding updaten}

\item \textbf{Demo}
  \begin{itemize}
   \item Wouter en Laurens zullen samenkomen om demo voor te bereiden
   \item \textbf{\emph{ZEER BELANGRIJK!!: iedereen moet zorgen dat zijn code bugvrij is tegen woensdag!}}
   \item Andere mensen mogen optioneel meegaan
   \item De volgende dingen moeten gevraagd worden aan Karel:
      \begin{itemize}
      	\item  http://edran.ugent.be/prod/ 

	\item  bij veranderen adres postcode en nr met een dropdown hoeft niet 
	\item  logout gaf een foutmelding

	\item  inschrijven voor een sessie gaf een foutmelding 

	\item  kon net aangemelde user en auto niet zien 

	\item  moeten we paginasysteem implementeren? 

	\item  wat zie je op 1 scherm ? (schermopbouw is vaak niet logisch)
      \end{itemize}
  \end{itemize}
  
\item \textbf{Gilles maakt nieuwe issues aan op github}


\par\item \textbf{Volgende vergadering} \\
Maandag 12 mei om 10:00
\end{itemize}

\end{document}
